\documentclass{article}

\title{PBE - Lecture 13}
\author{Ossama Edbali}

\begin{document}
	
	\maketitle
	
	\section*{Specialist planning controls}	
	In this session we will cover:
	\begin{itemize}
		\item Conservation areas
		\item Listed Buildings
		\item Planning gain
		\item Urban design
	\end{itemize}
	
	\subsection*{Conservation areas}
	LPAs have the duty to designate which areas must be protected from future intensive development. These
	areas may include historical spaces, areas of special architecture etc\ldots
	There are over 9000 conservation areas in England.
	
	\subsection*{Listed buildings}
	These are buildings of special or architectural value. For these 
	buildings have legal protection from demolition or alteration.
	There are various categories of listed buildings:
	\begin{itemize}
		\item \textbf{Grade 1}: buildings of exceptional interest (usually pre-1700)
		\item \textbf{Grade 2*}: buildings are particularly important buildings of more than special interest
		\item \textbf{Grade 2}: buildings are nationally important and of special interest
		\item \textbf{Local listing}: could be included in the national list 
	\end{itemize}
	
	\subsection*{Planning obligations}
	They were created under section 106 of TCPA 1990.
	Planning obligations under Section 106 of the Town and Country Planning Act 1990 (as amended), commonly known as s106
	agreements, are a mechanism which make a development proposal acceptable in planning terms, that would not otherwise
	be acceptable. They are focused on site specific mitigation of the impact of development. S106 agreements are often 
	referred to as 'developer contributions' along with highway contributions and the Community Infrastructure Levy.
	
	
	
\end{document}