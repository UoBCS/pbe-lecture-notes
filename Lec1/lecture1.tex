\documentclass{article}

\title{PBE - Lecture 1}
\author{Ossama Edbali}

\begin{document}
	
	\maketitle	
	
	\section{What is planning}
	Planning is a public-led (mainly) activity that deals with:
	\begin{itemize}
		\item Control of development in urban and rural areas
		\item Forward planning (projection, planning for the future)
		\item Fixing past issue/problems, urban regeneration
		\item Design of development (building-building and building-open space relationships)
		\item Place making and place shaping
		\item Controlling land use in the public interest
		\item Guidances on quality of housing (normative point of view)
	\end{itemize} 
	
	Planning has been defined as a balancing act between physical, social and economic factors as well
	as between public and private interest.	
	
	\subsection*{Useful definitions}
	\begin{description}
		\item[LPA] Local Planning Authority is a LA or council that is empowered by law to exercise statutory
		town planning functions for a particular area in the UK.
		\item[TCPA] Town and Country Planning Association is an independent charity in the UK. It works
		to improve town and country planning.
		\item[Green paper] is a consultation document produced by the Government. Its aim is to let people
		to discuss a particular issue both inside and outside the Parliament. 
		\item[White paper] is a document produced by the Government setting out details
		of future policy on a particular subject.
		\item[RSS] Regional Spatial Strategies
		\item[RES] Regional Economic Strategies
		\item[Local Development Framework] is the spatial planning strategy introduced in England by the
		Planning and Compulsory Purchase Act of 2004.
	\end{description}
	
	\section{Political context}
	In order to understand the policies of modern urban planning we must cover the political context.
	
	
	\section{The English Planning System (pre 2011)}
	(see lecture notes diagram)
	
	\section{The New English Planning System (post 2011)}
	(see lecture notes diagram)
	
\end{document}