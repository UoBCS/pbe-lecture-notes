\documentclass{article}

\usepackage{hyperref}
\title{PBE - Lecture 1}
\author{Ossama Edbali}

\begin{document}
	
	\maketitle	
	
	\section{What is planning}
	Planning is a public-led (mainly) activity that deals with:
	\begin{itemize}
		\item Control of development in urban and rural areas
		\item Forward planning (projection, planning for the future)
		\item Fixing past issue/problems, urban regeneration
		\item Design of development (building-building and building-open space relationships)
		\item Place making and place shaping
		\item Controlling land use in the public interest
		\item Guidances on quality of housing (normative point of view)
	\end{itemize} 
	
	Planning has been defined as a balancing act between physical, social and economic factors as well
	as between public and private interest.	
	
	\subsection*{Useful definitions and names}
	\begin{description}
		\item[LPA] Local Planning Authority is a LA or council that is empowered by law to exercise statutory
		town planning functions for a particular area in the UK.
		\item[TCPA] Town and Country Planning Association is an independent charity in the UK. It works
		to improve town and country planning.
		\item[Green paper] is a consultation document produced by the Government. Its aim is to let people
		to discuss a particular issue both inside and outside the Parliament. 
		\item[White paper] is a document produced by the Government setting out details
		of future policy on a particular subject.
		\item[RSS] Regional Spatial Strategies
		provided regional level planning frameworks for the regions of England outside London (2004)
		\item[RES] Regional Economic Strategies
		\item[Local Development Framework] is the spatial planning strategy introduced in England by the
		Planning and Compulsory Purchase Act of 2004.
		The local development framework replaces the previous system of county level structure plans and
		district level local plans, and unitary development plans for unitary authorities.
		The previous system was perceived as being too inflexible and difficult to change in a timely manner.
		The local development framework system is intended to improve this situation by replacing the old
		plans with a new portfolio of local development documents that can be tailored to suit the different
		needs of a particular area and can be easily updated.
		\item[Eric Pickles] current Secretary of State for Communities and Local Government.
	\end{description}
	
	\section{Road to reform}	
	The modern planning system is the result of various policies, guidances and acts that evolved
	in time.
	
	\subsection*{Planning Green Paper 2001}	
	The PGP 2001 set out some important goals:
	\begin{itemize}
		\item Simplify complex development plans
		\item Increase speed of plan production
		\item Better engagement with the community
		\item Abolition of structure plans
		\item Regional Spatial Strategy Documents replace current regional guidance and have statutory status
		\item Local Development Frameworks (LDFs) to replace local plans and unitary development plans
	\end{itemize}		
	
	\subsection*{Planning and Compulsory Act 2004}	
	The Planning and Compulsory Purchase Act 2004 is a key element of the Government’s agenda for speeding up the planning system. The provisions introduce powers which allow for the reform and speeding up of the plans system and an increase in the predictability of planning decisions, the speeding up of the handling of major infrastructure projects and the need for simplified planning zones to be identified in the strategic plan for a region or in relation to Wales. The Act also provides for a number of reforms to make the handling of planning applications by both central government and local authorities quicker and more efficient. There are also provisions to make the planning Acts bind the Crown, fulfilling a long-standing commitment to end the Crown's immunity from the planning system. The provisions relating to compulsory purchase powers and compensation will liberalise the compulsory purchase and compensation regimes. They support policies relating to investment in major infrastructure and regeneration. (\url{http://www.legislation.gov.uk/ukpga/2004/5/notes/division/2})
	
	The main provisions:
	\begin{itemize}
		\item Structure plans abolished
		\item RPG $\rightarrow$ RSS
		\item Flexible local plan policy $\rightarrow$ LDF
		\item Wider compulsory purchase powers
	\end{itemize}
	
	\subsection*{Barker Review 2006}	
	\begin{itemize}
		\item Introduce new system for dealing with major infrastructure projects
		\item Promote positive planning culture
		\item More risk-based and proportionate approach to regulation
	\end{itemize}
	
	\subsection*{Planning White Paper 2007}
	The Planning White Paper sets out the detailed proposals for reform of the planning system,
	building on Kate Barker's recommendations for improving the speed, responsiveness and efficiency
	in land use planning, and taking forward Kate Barker's and Rod Eddington's proposals
	for reform of major infrastructure planning.	
	
	It also proposes further reforms to the Town and Country Planning system,
	building on the recent improvements to make it more efficient and more responsive.	
	It also encourages community involvement.	
	
	\subsection*{Planning Act 2008}		
	It is intended to speed up the process for approving major new infrastructure
	projects such as airports, roads, harbours, energy facilities such as nuclear power and waste facilities.	
	
	\subsection*{Localism Act 2011}	
	The aim of the act is to facilitate the devolution of decision-making powers
	from central government control to individuals and communities	 (more power to local govs).
	
	Main points:
	\begin{itemize}
		\item Neighbourhood involvement
		\item Sustainable development
		\item Abolition of RSSs but there is a duty for
		interested parties to co-operate in the preparation of development plans
		\item Greater financial autonomy to local government and community groups
		\item Open source planning
		\item Measures to bring empty homes into use
	\end{itemize}
	
	\section{Political context}
	In order to understand the policies of modern urban planning we must cover the political context.
	
	
	\section{The English Planning System (pre 2011)}
	(see lecture notes diagram)
	
	\section{The New English Planning System (post 2011)}
	(see lecture notes diagram)
	
\end{document}