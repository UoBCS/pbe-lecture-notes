\documentclass{article}

\title{PBE - Lecture 5}
\author{Ossama Edbali}

\begin{document}
	
	\maketitle
	
	\section*{New Right Critique}
	Alongside the Marxist critique there is the new right critique to the positivist movement.
	They were ideologically committed to libertarianism as well as being socially conservative.
	Key policies included deregulation of business, a dismantling of the welfare state,
	privatization of nationalized industries and restructuring of the national workforce
	in order to increase industrial and economic flexibility in an increasingly global market.
	
	The context in which the new right movement began to spread was the failure of the labour government:
	unemployment, post-war planning focussed on modernism, high levels of bureaucracy.
	
	The key themes of the new right critique are:
	\begin{itemize}
		\item Limited government influence
		\item Class division is the natural process (punishment - reward)
		\item Reduce bureaucracy
	\end{itemize}
	
	Friedrich Hayek was one of the most known liberal thinkers. His view is:
	\begin{itemize}
		\item Central planning is dangerous and inefficient
		\item Limited role of government
		\item Local planning desirable to support market
	\end{itemize}
	
	Thatcherism was the philosophy in the 70s:
	\begin{itemize}
		\item Public sector was too big and complex
		\item Private sector is better and provides economic growth and more efficient
		\item Promotion of enterprise, initiative and self-reliance
	\end{itemize}
	
	The key changes in planning in the Thatcher era were:
	\begin{itemize}
		\item Speed of planning processes
		\item Focus on land use
		\item Economic growth
		\item Participation
		\item Enterprise zones
	\end{itemize}
	
	\section*{Enterprise zones}
	An Urban Enterprise Zone is an area in which policies to encourage economic
	growth and development are implemented.
	Urban Enterprise Zone policies generally offer tax concession, infrastructure incentives,
	and reduced regulations to attract investments and private companies into the zones.
	Urban Enterprise Zones are common in the United Kingdom and the United States.	
	
	\section*{Urban Development Corporations}
	Development corporations are bodies set up in England and Wales by the UK government charged
	with the urban development of an area, outside the usual system of Town and Country Planning
	in the United Kingdom. Members are appointed by central government and hence they are considered QUANGOs
	(quasi-autonomous non-governmental organisation).
	
\end{document}