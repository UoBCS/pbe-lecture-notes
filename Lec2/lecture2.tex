\documentclass{article}

\title{PBE - Lecture 2}
\author{Ossama Edbali}

\begin{document}
	
	\maketitle
	
	\section{Why plan?}
	Planning is a fundamental activity for the cities and towns to be better places to live in.
	These are some reasons why planning is an important activity:
	\begin{itemize}
		\item Controlling city/town growth
		\item Protect natural (sensitive habitats) and historical areas
		\item Urban regeneration (e.g. Birmingham city centre)
		\item Prevent mistakes such as out of town shopping centres
		\item Public betterment (policies and guidances)
		\item Controlling quality of life
		\item Transport (reduce the need to travel)
		\item Maximise the use of land
		\item Sustainable development $\rightarrow$ best use of land
		\item Forward-looking
		\item Promote community development
	\end{itemize}
	
	\section{Key planning themes}
	There are 3 key planning themes:
	\begin{itemize}
		\item Sustainable development
		\item Place-making and shaping
		\item Community involvement
	\end{itemize}
	
	\subsection*{Sustainable development}
	There are 3 key entities in sustainable development: economy, society and environment.
	Sustainable development tries to meet the nation's needs whilst respecting environmental
	policies or objectives.
	Therefore it maximises the use of land as well as using already developed areas. The consequence of
	such approach is the conservation of both cultural heritage and natural resources.
	
	A good example of sustainable development are the so-called sustainable communities:
	\begin{quote}
		Sustainable communities are places where people want to live and work, now and in the future.
		They meet the diverse needs of existing and future residents, are sensitive to their environment,
		and contribute to a high quality of life. They are safe and inclusive, well planned, built and run,
		and offer equality of opportunity and good services for all.
	\end{quote}
	\textit{Homes and Communities Agency}
	
	\subsection*{Place-making and shaping}	
	The key factors in place-making are:
	\begin{itemize}
		\item Governance
		\item Transport and connectivity
		\item Services
		\item Environment
		\item Equity (equal opportunities)
		\item Economy
		\item Housing and built environment
		\item Social and cultural
	\end{itemize}		
	
	\subsection*{Community involvement}	
	Planning affects everyone therefore there must be a huge community involvement in planning committees,
	meetings and other activities. Community involvement make the process of planning transparent,
	accessible, fair and actively promote participation.
	
	With the coalition agreement (2010) the planning policy waived from centralisation, regional strategies
	and focussed more on communities and local power.
	
	
\end{document}