\documentclass{article}

\title{PBE - Lecture 6}
\author{Ossama Edbali}

\begin{document}
	
	\maketitle
	
	\section*{Postmodernity and planning}	
	Postmodernism is a rejection of 'totality', of the notion that planning could be 'comprehensive', widely applied regardless of context, and rational. In this sense, Postmodernism is a rejection of its predecessor: Modernism. From the 1920s onwards, the Modern movement sought to design and plan cities which followed the logic of the new model of industrial mass production; reverting to large-scale solutions, aesthetic standardisation and prefabricated design solutions (Goodchild 1990). Postmodernism also brought a break from the notion that planning and architecture could result in social reform, which was an integral dimension of the plans of Modernism (Simonsen 1990).
	
	The key theorists in the PoMo movement are Jean Francois Lyotard, Michel Foucault, Jean Baudrillard.
	
	\section*{Collaborative planning}
	Collaborative planning based on communicative rationality means to understand the world through
	free and open dialogue (based on Habermas thinking).
	In collaborative planning the city is brought to life by a collective process of imagining what it is
	and what it will be.
	
	A huge contribution in this way of thinking is from Sandercock (1998) \textit{Towards Cosmopolis}.
	
	\section*{The critique of Modernist planning}
	The 5 pillars of the modernist planning are:
	\begin{itemize}
		\item Rational
		\item Comprehensive
		\item Science and art
		\item Progressive, reformist
		\item Public interest
	\end{itemize}
	
\end{document}