\documentclass{article}

\title{PBE - Lecture 3}
\author{Ossama Edbali}

\begin{document}
	
	\maketitle
	
	\section{Positivism and planning}	
	Positivism is the philosophy driven by scientific methods, empirical proving and formulation
	of universal laws.	
	
	Positivism has to a certain extent, an impact on urban planning.
	Looking at the positivist view of the world, urban planners can have a certain influence on the urban form
	of the world looking at the fact that in a positivist world, science was seen as the way to get at truth, to
	understand the world well enough so that we might predict and control it. Therefore an urban planner
	assuming this position would tend to approach planning mainly from the technical aspect of it and not
	necessarily engage the society in the planning process and therefore planning would be done from a top-down
	approach and public participation would be very minimal.
	
	The context in which positivist urban planners worked was the post-war reconstruction process.
	There was a need of comprehensive redevelopment as well as slum clearance. Their view influenced the
	modern movement:
	\begin{itemize}
		\item Breaking with the past
		\item Comprehensive planning
		\item Ordered city
		\item Linear and rigorous planning
		\item New materials and technology
	\end{itemize}
	
	Le Corbusier was one of the most famous urban planners (such as Antonio Sant'Elia)
	who was a pioneer in the modern movement.
	
	However positivist urban planning was too idealist, the scientific method is not appropriate for
	social science.
	
\end{document}